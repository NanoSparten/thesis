\documentclass{article}[a4paper]
\usepackage[pdftex]{hyperref}
\usepackage[backend=bibtex,sorting=none]{biblatex}
\addbibresource{refs}

\newcommand{\bitem}[2]{
    \item[\cite{#1}]
        \citetitle{#1}, \citeyear{#1}
        \newline
        \citeauthor{#1}{ #2}
}

\begin{document}

\section{Feature Extraction}
\begin{itemize}
    % GLCM
    \bitem{haralick_textural_1973}{describe an easily computable feature set for
        texture analysis called gray level co-occurrence matrices (GLCM). GLCM
        shows the distribution of co-occurring pixel pairs given a certain
        distance and direction. Their results show that the method can be
        generally applicable for image classification and especially for texture
        analysis.}

    % LBP
    \bitem{ojala_multiresolution_2002}{propose a feature
        extraction method that uses gray-scale pixel values and is shown to be
        rotation invariant for classification purposes. The proposed method (LBP) is
        also computationally efficient and can be used for texture analysis. The
        method is grayscale invariant because the neighboring pixels are scaled
        based on the center pixel value without losing any information. The
        method is also rotation invariant because the neighboring pixels are
        related to the center pixel by a rotation-based relationship.}

    % SIFT
    \bitem{lazebnik_local_2006}{}

    % SPIN
    \bitem{zhang_local_2007}{}

    \bitem{xu_integrating_2009}{propose a texture descriptor based on
        \cite{lazebnik_local_2006} and \cite{zhang_local_2007}. They integrate
        the local-affine feature descriptors from \cite{lazebnik_local_2006} and
        \cite{zhang_local_2007} along with a global statisical feature called
        multi-fractal spectrum (MFS). By combining the two features, they create
        a feature descriptor that is robust to both illumination and affine
        transforms.}

    % MDLBP
    \bitem{bedi_mean_2021}{propose Mean distance LBP (MDLBP) that extends
        \cite{ojala_multiresolution_2002}. Their method changes the relationship
        between the neighboring pixels and the center pixels such that the
        nearest pixels are given a higher weight than farther pixels and
        therefore have a stronger influence on the output pattern.}
\end{itemize}

\section{Algorithms}
\begin{itemize}
    % Paper dating analysis
    \bitem{lu_paper_2021}{propose a method of texture analysis to date paper
        from ancient books. They describe using a CNN to first extract global
        texture features and using a hybrid attention model to extract local
        texture features. Then, they combine the features as input to a GRU
        (Gated Recurrent Unit) to train a paper dating time-series model. Their
        results show comparisons of different loss functions when training their
        model.}
\end{itemize}

\printbibliography
\end{document}
